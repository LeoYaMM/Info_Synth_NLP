\documentclass{report}
\usepackage[utf8]{inputenc}
\usepackage[spanish]{babel}
\usepackage[margin=2cm]{geometry}
\usepackage{graphicx}
\usepackage{float}
\usepackage{titlesec}
\usepackage{caption}
\usepackage{listings}
\usepackage{xcolor}
\usepackage{array}
\usepackage{booktabs}
\usepackage{tabularx}
\usepackage{multirow}
\usepackage{amsmath}
\usepackage{hyperref}
\usepackage{ragged2e} 
\usepackage{lipsum}

\definecolor{codegreen}{rgb}{0,0.6,0}
\definecolor{codegray}{rgb}{0.5,0.5,0.5}
\definecolor{codepurple}{rgb}{0.58,0,0.82}
\definecolor{backcolor}{rgb}{0.95,0.95,0.95}

\lstset{
    basicstyle=\ttfamily,
    inputencoding=utf8,
    extendedchars=true,
    literate=%
    {á}{{\'a}}1
    {é}{{\'e}}1
    {í}{{\'i}}1
    {ó}{{\'o}}1
    {ú}{{\'u}}1
    {ñ}{{\~n}}1
    {Á}{{\'A}}1
    {É}{{\'E}}1
    {Í}{{\'I}}1
    {Ó}{{\'O}}1
    {Ú}{{\'U}}1
    {Ñ}{{\~N}}1
}

\lstdefinestyle{mystyle}{
    backgroundcolor=\color{backcolor},
    commentstyle=\color{codegreen},
    keywordstyle=\color{red},
    numberstyle=\tiny\color{codegray},
    stringstyle=\color{codepurple},
    basicstyle=\ttfamily\footnotesize,
    breakatwhitespace=false,
    breaklines=true,
    captionpos=b,
    keepspaces=true,
    numbers=left,
    showspaces=false,
    showstringspaces=false,
    showtabs=false,
    tabsize=2  
}

\titleformat{\section}
{\huge\bfseries}{\thesection.}{1em}{}
\titleformat{\subsection}
{\large\bfseries}{\thesubsection}{1em}{}

\renewcommand\thesection{\arabic{section}}

\title{\Huge{\textbf{Modelo de Procesos y Modelo de Datos}}\\
\Large{\textbf{Ingeniería de Software Para Sistemas Intelgentes}}}
\author{Diego Castillo Reyes\\Marthon Leobardo Yañez Martinez\\Aldo Escamilla Resendiz}

\graphicspath{{imagenes/}}

\begin{document}
\maketitle
\section{Modelo de Procesos}
\subsection*{Diagrama de Secuencia}
A continuación de mostrará el diagrama de secuencia donde se
muestra el proceso del sistema:
\begin{figure}[H]
    \centering
    \includegraphics[width=0.8\textwidth]{Diagrama de secuencia.png}
    \caption{Diagrama de Secuencia}
\end{figure}
\subsection*{Descripción de Procesos}
\begin{itemize}
    \item Inicio del recorrido: El usuario ingresa su edad, el sistema guarda el dato y le permite escanear el QR de inicio de sala.
    \item Recorrido interactivo: A medida que el usuario escanea los QR de la exposición, el sistema Thot genera resúmenes personalizados que se muestran en la interfaz del usuario.
    \item Finalización del recorrido: El usuario escanea el QR de salida, el servidor valida el escaneo y elimina los datos del usuario. Finalmente, el sistema muestra un mensaje de despedida al usuario.
\end{itemize}
\end{document}